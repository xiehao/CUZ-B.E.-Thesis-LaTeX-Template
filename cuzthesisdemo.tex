% !Mode:: "TeX:UTF-8"
% !TEX builder = LATEXMK
% !TEX program = xelatex
\documentclass[oneside,UTF8,AutoFakeBold=2]{cuzthesis} % 如果你的论文不满80页,还是单面印刷吧


%%%%%%%%%%%%%%%%%%%%%%%%%%%%%% 开始填写前置部分使用的变量
%%%%%%%%%%%%%%%%%%%%%%%%%%%%%% 样式设定在 zjuthesis.cls 下, 人类可读,爱请查阅

% 这里写这么鬼畜是为了测试多几个字会不会造成溢出
\title{一个非常非常长的必须分为2行的\LaTeX 论文模板标题} % 封面和题名页使用
\englishtitle{A Very Very Long Title of a \LaTeX\ Thesis Template with 2 Lines} % 封面和题名页使用
% 如果您的标题用字过多,请自行调节 zjuthesis.cls 里的 ZJUmakecover 里的各项距离。

\author{张无忌}              % 学生姓名
\graduate{20XX}				% 届
\studentnumber{150XXXXXX}   % 学号
\supervisor{张三丰}          % 指导教师
\supervisortitle{教授}       % 职称
\major{数字媒体技术}		   % 专业
\institute{新媒体学院}		% 所在学位栏 填 软件学院

% 论文前置部分变量填写完毕 开始全书排版
\begin{document}

% 封面、中文题名页、英文题名页、独创声明和版权使用书 无页码
\maketitle

% 摘要部分
\abstractmatter
\include{contents/abstract_chinese}
\include{contents/abstract_english}

% 目录和术语表
\frontmatter
\tableofcontents % 正文目录
\listoffigures   % 图目录
\listoftables    % 表目录
% 术语及缩略词表(需要则开)
%\include{contents/denotation}

% 正文排版开始 建议一章一文件 (好像无法嵌套 include) 
\mainmatter
\include{contents/intro} % 绪论
\include{contents/whyla}
\include{contents/elem}
\include{contents/sum}  % 总结和展望

% 结尾部分排版
\backmatter

% 引用参考文献数据库
\bibliography{references/test.bib}

% 附录部分
%\appendix
%\include{contents/appendixA}

% 作者简历
\include{contents/self}

% 致谢
% 致谢不必感谢在下,
% 但请一定感谢清华大学薛瑞尼、
% 机械工程学院陈九历
% !TEX root = ../main.tex
\chapter{致\CUZspace{}谢}

在去年撰写开题报告和文献综述时,
搜索Github发现了仅在数月前发布的ZJU-Awesome项目。
简单套用后自觉此法应当推广。
于是在2016年3月撰写论文结束后,
下决心向软件学院的同学推广此一模板,
继而有了本文和在下对模板的调整。

衷心感谢ZJU-Awesome项目作者之一机械工程学院的陈九历同学耐心解答在下的多次疑问。

% 难道不应该感谢世界(大雾)
也感谢夏目友人帐能让我在论文写作过程中保持着一份平静,
最后感谢这份文稿前的你,能仔细阅读到这一页。

\vspace{2cm}
\hfill
\begin{minipage}{14em}
    \begin{flushright}
        君之名\\
        %于浙江大学软件学院\\ % 学院要求的格式 - -#
        %2016年4月18日   % 与封面论文提交时间一致
        2017年1月1日\\
        于老和山下
    \end{flushright}
\end{minipage}


\end{document}
